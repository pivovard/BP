\chapter{Testování}

Od aplikace je vyžadována vysoká spolehlivost. Proto musí být důkladně otestována. Testováná je základní funkcionalita (tj správné chování aplikace, konzistentnost dat v databázi) a chování při zadání nekorektních dat (data mimo povolený rozsah, data v nesprávném formátu). Testování je prováděno manuálně.

První fáze testování probíhá již při vytváření nové funkcionality. Já a kolega \emph{Daniel Švarc}, který se podílí na vývoji aplikace (logická část a přístup do databáze), testujeme správnost nově vytvořené funkcionality a rekci na zadání nekorektních dat.

Druhá fáze probíhá před vytvořením release verze. V této fázi probíhá test celé aplikace. Především je kladen důraz na otestování všech nově přidaných funkcionalit a těch částí aplikace, které jimy mohly být ovlivněny. První a druhá fáze testování probíhá na zkušební databázi mimo FN Plzeň.

Třetí fáze probíhá na oddělení SIS\footnote{Správa informačního systému} ve FN Plzeň na jejich zkušební databázi. Testování provádí kolegové ze SIS, kteří testují především nové funkce a celkovou funkčnost aplikace.

Čtvrtou fází je pilotní provoz na JIP II. IK\footnote{Jednotka intenzivní péče II. interní kliniky} FN Plzeň. Aplikace je nasazena do ostrého provozu na reálné databázi. Zdravotní sestry testují správnost chování aplikace během běžného provozu a přívětivost uživatelského rozhraní. První pilotní provoz proběhl v průběhu února a března 2016. Během provozu bylo opraveno hodně chyb. Přesto se musel pozastavit, protože některé funkce aplikace nebyly vyhovující a přišel požadavek na přídání dalších funkcí. Druhý pilotní provoz je plánován začít v červnu nebo červenci 2016.