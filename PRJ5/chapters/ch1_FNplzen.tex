\setlength{\parskip}{1em}

\chapter{Prostředí FN Plzeň}

Tato kapitola se zabývá obecným popisem prostředí jednotky intenzivní péče ve FN Plzeň, databáze ve FN Plzeň a aplikace WinMedicalc - karet \emph{Ordinované léky}, \emph{Bilance tekutin} a \emph{Invazivní přístupy}.

\section{Práce zdravotních sester}

Práce zdravotních sester je ve stresujícím prostředí. Obvzlášť tomu je na jednotce intenzivní péče, kde se musí rychle reagovat na změny stavu pacientů a kde se nesmí dělat žádné chyby. Tomu musí být přizpůsobená i vyvíjená aplikace. Její ovládání musí být jednoduché a intuitivní, aby práce s ní byla efektivní.

Zdravotní sestry na jednotce intenzivní péče často pracují v latexových rukavicích. Tomu by mělo odpovídat i uživatelské rozhraní. Jednotlivé komponenty by měly proto být dostatečně veliké, aby nedocházelo ke zbytečným překlepům. Je třeba počítat i s tím, že některá zdravotní sestra může mít zrakovou vadu. Neustálé nasazování brýlý by ji poté zdržovalo od práce. Také proto je nutné použít dostatečně velké komponenty a písmo.

\section{WinMedicalc}

WinMedicalc je nemocniční informační systém, který usnadňuje a zrychluje vytváření lékařské dokumentace. Dále zajišťuje vykazování zdravotní péče a uchovávání dat v jednotné struktuře. Také obsahuje nástroje z oblasti managementu.

Každý pracovník FN Plzeň má osobní přístup této aplikace s povolénými funkcemi vzhledem k pracovní pozici.

\subsection{Ordinované léky}

Na kartě ordinovaných léků lékař zadává léky, které se se pacientovi mají podávat. Ke každému léku doplňje množství, kolikrát denně a od kdy do kdy se má lék podávat (viz obrázek \ref{fig:WM_ordinovane_leky}).

\begin{figure}[h]
	\centering
	%\includegraphics{../img/medicalc/WM_ordinovane_leky.JPG}
	\caption{Ordinované léky}
  \label{fig:WM_ordinovane_leky}
\end{figure}

Medikační karta se poté vytiskne na papír do tabulky s vynačenými hodinami. Tabulka nezačíná od půlnoci, ale od hodiny, kterou mají na oddělení nastavenou jako začátek dne (obvykle to bývá 6-7 hodina).

Medikační kartu musí vždy před vytištěním schválit lékař. Pokud to nestihne do začátku dne, vytiskne se karta z předchozího dne. Zdravotní sestry pak podávají léky podle této karty do doby, než se schválí a vytiskne nová medikační karta. Překrývající se údaje poté přepíší.

\subsection{Bilance tekutin}

V bilanci tekutin se zaznamenává příjem a výdej veškerých tekutin pacienta na lůžku i na operačním sále za celý den. Zaznamenává se 7 tekutin pro příjem a 5 tekutin pro výdej. Zároveň se počítá celkový příjem a celkový výdej všech tekutin. Rozložení jednotlivých tekutin je vidět na obrázku \ref{fig:WM_bilance_tekutin}.

\begin{figure}[h]
	\centering
	%\includegraphics{../img/medicalc/WM_bilance_tekutin.JPG}
	\caption{Bilance tekutin}
  \label{fig:WM_bilance_tekutin}
\end{figure}


Příjem a výdej tekutin se odečítá během dne několikrát, ke každé tekutině tedy během dne bude několik hodnot. Všechny hodnoty se ukládají do databáze, aby bylo možné sledovat jejich vývoj.

Zdravotní sestry zaznamenávají příjem a výdej tekutin na papír. Na konci dne přepíší údaje do WinMedicalcu.

\subsection{Invazivní přístupy}

Každý pacient může mít zaveden libovolný počet různých invazivních přístupů. Každý přístup má vlastní specifikaci (číslo, název, umístění, hloubku zavedení, datum zavedení a počet dnů zavedení). Určité přístupy lze nechat zavedeny v pacientovy pouze po určitý počet dní. Proto každý přístup lze označit požadavkem na výměnu nebo jako vyměněný. Také lze přístup odebrat pokud byl pacientovi vyndán. Rozložení karty přístupů je na obrázku \ref{fig:WM_invazivni_pristupy}.

\begin{figure}[h]
	\centering
	%\includegraphics{../img/medicalc/WM_invazivni_pristupy.JPG}
	\caption{Invazivní přístupy}
  \label{fig:WM_invazivni_pristupy}
\end{figure}

Zdravotní sestry mají vytištěn seznam invazivních přístupů, ve kterém si značí změny. Na konci dne přepíší provedené zmeny do WinMedicalcu.

\subsection{Databáze}

Ve FN Plzeň je rozsáhlá databáze od Oraclu. V ní se ukládají prakticky všechny záznamy, včetně záznamů o pracovnících FN Plzeň, záznamů o pacientech a klinických událostech. Tato data se využijí ve vyvíjené aplikaci.

Jelikož přístup do databáze je pouze ve FN Plzeň, byla její část zkopírována do nového tablespace v Oracle databázi na ZČU.